\documentclass[11pt,a4paper,uplatex]{jsarticle}
\usepackage{url}

\begin{document}

\section{概要}
\begin{itemize}
    \item 研究日誌: 研究の進捗等のメモ
\end{itemize}

\section{2025/04/18}
\begin{itemize}
    \item 初めてRaspberryPiを触る. なにがなんだかわからないがとりあえずishisakiさんを頼る.

    \item OSの入ったSDカードがなかったので, SDカードリーダを使って自分のPCからRaspi用のOSをダウンロード.
    \item その後, MHZ19C(CO$_2$濃度センサ)とのシリアル通信を試す. \url{}
    \begin{itemize}
        \item PWM: Chatgptにpythonプログラムの雛形を作ってもらい. 詳細部を調整してシリアル通信ができた.
        \item UART: 苦戦した. シリアル通信を行うためには, ポートでセンサなどのデバイスを認識する必要があるのだが, その部分の設定がうまくいかない. \\
        結局, ishisakiさんが解決してくださった. 不具合の原因は, デフォルトで設定されていたポートのGPIOがLinuxターミナルですでに使われているのに使おうとしていたため. 
        ターミナルから他のポートを有効化してそのポートを設定することで解決.
    \end{itemize}
    \item 結論: ishisakiさん is God.
\end{itemize}


\section{2025/04/21}
\begin{itemize}
    \item 他のCO$_2$濃度センサを試すことにした.(筒井先生おすすめのセンサ??) Senseair Sunriseというもの. \url{}
    \item またしても大苦戦. 配線でつなぐとこまではQiitaの記事などを参考に完了.(Lチカで随時確認)
    \item SenseairはModbusという通信プロトコルでUARTを行うそうだが, Modbusの仕様書が英語で理解に苦しむ.
    \item 遅くなったので帰宅. 帰宅途中の電車で日本語のModbus仕様書を発見. 仕様書のコマンド部分を見ると何も難しくなさそう. 明日研究室で試そう.
    \item 結論: 英語はクソ
\end{itemize}


\section{2025/04/22}
\begin{itemize}
    \item B4輪講, ゼミのためスタートが遅くなった. M2の先輩方が周りで何かに悲観している$\dots$
    \item 昨日見つけた仕様書の通りにプログラムを作成. 正しそうなppmをターミナルにて確認(500ppm)ぐらい.
    \item 結論: 英語の勉強をしよう.
    \item 参考資料
    \begin{itemize}
        \item Modbusプロトコル概説書: \url{https://www.mgco.jp/mssjapanese/PDF/NM/kaisetsu/nmmodbus.pdf}
        \item Modbus on Senseair Sunrise: \url{https://senseair.jp/wp-content/uploads/2019/01/TDE5514.pdf}
        \item Senseair Sunrise Product Specification: \url{https://rmtplusstoragesenseair.blob.core.windows.net/docs/Dev/publicerat/PSP12440.pdf}
    \end{itemize}
\end{itemize}


\section{2025/04/23}
\begin{itemize}
    \item 今日は研究室に誰もいなかった. 昨日悲観していた先輩方は無事だろうか$\dots$
    \item Lora通信モジュールをRasPiに取り付けて通信したい.(もともとの目的)
    \begin{itemize}
        \item 通信モジュールをishisakiさんの所から拝借. USBE220-900T22S(黒): USBでつなぐタイプ
        \item ポートの設定などはだいぶ慣れてきた. 仕様書とにらめっこを開始. 今日は勝てる気がする
        \item 仕様書によるとこのモジュールには大きく4つのModeがあるらしい.(手動切り替え) まずMode3にてデバイスのアドレスやらLoraのパラメータやらを設定する必要がある. \\ 
        RasPi側のシリアル通信にて正しく設定できたことを確認した.(requestとresponseにて) 受信側は研究室のPCを用いた(こちらも同様に設定) \\
        送信用コードを用意("Hello"を送信するだけの簡単なもの) 受信用コードも用意→通信を行うと受信側でHelloの文字を観測.
    \end{itemize}
    \item 結論: もう怖いものは何もない.
    \item 参考資料
    \begin{itemize}
        \item Lora通信モジュール使い方, 特にUSB型(黒)専用の説明: \url{https://flint.works/p/flint-lora-usb/}
        \item Lora通信のデータシート(パラメータやコマンド説明など): \url{https://dragon-torch.tech/wp-content/uploads/2024/09/DS240911JA_E220-900T22Sv1_Rev2.1.1.pdf}
        \item 同じくデータシート(簡潔): \url{https://dragon-torch.tech/wp-content/uploads/2022/08/data_sheet.pdf}
    \end{itemize}
\end{itemize}

\section{2025/04/24}
\begin{itemize}
    \item 今日も研究室に誰もいない. 「工学部のすべて」の写真撮影を終えた後に作業開始.
    \item 


\end{document}