documentclass[article]

\begin{document}

\section{2025/05/12}
とりあえずデータ取得開始.


\section{2025/05/14}
\begin{itemize}
    \item LoRa: 通信間隔1分
    \item 10:57送信開始. 電池残量4.17V ⇛ 20:19まで正常に動作を確認. (9時間22分)
\end{itemize}

\section{2025/05/19}
\begin{itemize}
    \item SDカードに保存: 保存間隔1分
    \item 16:13頃開始. 電池残量4.19V.
\end{itemize}

\section{2025/05/20}
\begin{itemize}
    \item LoRa: 通信間隔2分
    \item 17:19送信開始. 電池残量4.18V ⇛ 02:02まで正常に動作を確認. (終了後電池残量3.04V)
\end{itemize}

\section{2025/05/21}
\begin{itemize}
    \item LoRa: 通信間隔1分, 送信を終えるたびにDeepSleepモードになるように変更.
    \item 10:35送信開始. 電池残量4.18V ⇛ 翌日14:00まで正常に動作. (27時間25分)
\end{itemize}

\section{2025/05/23}
\begin{itemize}
    \item LoRa: 通信間隔2分, DeepSleep
    \item 11:11送信開始, 電池残量4.19V ⇛ 翌日14:33まで正常に動作.  (終了時電圧電池残量 2.52V(05/26/10:53))
\end{itemize}

\section{2025/05/26}
\begin{itemize}
    \item LoRa: 通信間隔10分, DeepSleep+センサ側もセンシングの度に停止.
    \item 15:20送信開始, 電池残量4.19V ⇛ 翌日20:01まで正常に動作 ()
    \item センサ側の停止はデフォルトでreadするときのみになっているらしいので, 単純に10分にしたことになっている(かも)
\end{itemize}

\section{2025/06/02}
\begin{itemize}
    \item マイコン: SLEEPDEEPモードで実行. 間隔1分.
    \item 16:08送信開始, 電池残量4.2V ⇛ 
\end{itemize}


\section{2025/06/03}
\begin{itemize}
    \item マイコンは一旦忘れる. LoRa: 通信間隔1秒. 
    \item 18:15送信開始, 電池残量4.2V ⇛ 翌日02:05まで動作. (7時間50分)
\end{itemize}

\section{2025/06/09}
\begin{itemize}
    \item パラメータを最適と思われるものに設定. 
    \item 1回の送信時間は60ms. 
    \item LoRa: 間隔1分
    \item 16:15送信開始, 電池残量4.19V ⇛ 翌日18:04まで動作. (25時間49分)
\end{itemize}

\section{2025/06/10}
\begin{itemize}
    \item LoRa: 間隔2秒
    \item 18:33送信開始, 電池残量4.2V ⇛ 翌日04:38まで動作. (10時間5分)
\end{itemize}

\section{2025/06/17}
\begin{itemize}
    \item LoRa: 間隔5秒
    \item 15:36送信開始, 電池残量4.19V ⇛ 翌日07:11まで動作. (15時間35分)
\end{itemize}

\section{2025/06/24}
\begin{itemize}
    \item LoRa: 間隔10秒
    \item 12:43送信開始, 電池残量4.19V ⇛ 翌日07:01まで動作 (18時間18分)
\end{itemize}

\section{2025/06/27}
\begin{itemize}
    \item LoRa: 間隔30秒
    \item 11:59送信開始, 電池残量4.19V ⇛ 翌日12:21まで動作 (24時間22分)
\end{itemize}

\section{2025/06/30}
\begin{itemize}
    \item LoRa: 間隔10秒(データ取り直し)
    \item 13:52送信開始, 電池残量4.19V ⇛ 翌日10:15まで動作 (20時間23分)
\end{itemize}

\section{2025/07/02}
\begin{itemize}
    \item マイコンのSleepモードを実装. WDT間隔10秒(受信間隔は約12秒)
    \item 16:16送信開始, 電池残量4.2V ⇛ 翌日13:35まで動作 (21時間19分)
\end{itemize}

\section{2025/07/04}
\begin{itemize}
    \item 10秒だとあまり差を感じられないため, WDT間隔を60秒に(プログラムの動作に約2秒かかる)
    \item 16:11送信開始, 電池残量4.2V ⇛ 翌日18:37まで動作 (26時間26分)
\end{itemize}

\section{2025/07/07}
\begin{itemize}
    \item モジュールを.end()してから, __WFI()を実行するように変更. これで無理ならもう原因がわかりません$\dots$
    \item 12:37送信開始, 電池残量4.2V ⇛ 変化なさそうなので変更. 間隔を120秒に
    \item 14:12送信開始, 電池残量4.19V ⇛ 翌日21:09まで動作 (30時間57分)
\end{itemize}

\section{2025/07/11}
\begin{itemize}
    \item 間隔10分(600秒)
    \item 15:53送信開始, 電池残量4.2V ⇛ 翌々日の01:12まで動作 (33時間19分)
\end{itemize}

\section{2025/07/14}
\begin{itemize}
    \item 間隔30分 (1800秒)
    \item 12:19送信開始, ⇛ 翌日の19:51まで動作 (31時間32分)
\end{itemize}

\section{2025/07/18}
\begin{itemize}
    \item もう一度間隔1分(60秒)
    \item 13:30送信開始 ⇛ 翌日の19:38まで動作 (30時間8分)
\end{document}

\section{2025/07/21}
\begin{itemize}
    \item 間隔10秒
    \item 14:46送信開始 → 翌日11:05まで動作 (20時間19分)
\end{itemize}

\section{2025/07/25}
\begin{itemize}
    \item NoSleepのdelay(1分)のデータをもう一度取る
    \item 15:15送信開始 → 翌日16:15まで動作 (25時間0分)
\end{itemize}
<<<<<<< HEAD

\section{2025/08/12}
\begin{itemize}
    \item LoRaのDeepSleepも用いないデータを取る(最初期のやつ)
    \item 間隔は10秒(すでに1分は取ってある)
    \item 15:48送信開始 → 当日23:43まで動作 (7時間55分)
\end{itemize}

\section{2025/08/16}
\begin{itemize}
    \item NoSleep間隔30秒
    \item 14:26送信開始 → 23:05まで動作 (8時間39分)
\end{itemize}

=======
>>>>>>> thesis

\end{document}

