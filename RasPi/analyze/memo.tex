documentclass[article]

\begin{document}

\section{2025/05/12}
とりあえずデータ取得開始.


\section{2025/05/14}
\begin{itemize}
    \item LoRa: 通信間隔1分
    \item 10:57送信開始. 電池残量4.17V ⇛ 20:19まで正常に動作を確認.
\end{itemize}

\section{2025/05/19}
\begin{itemize}
    \item SDカードに保存: 保存間隔1分
    \item 16:13頃開始. 電池残量4.19V.
\end{itemize}

\section{2025/05/20}
\begin{itemize}
    \item LoRa: 通信間隔2分
    \item 17:19送信開始. 電池残量4.18V ⇛ 02:02まで正常に動作を確認. (終了後電池残量3.04V)
\end{itemize}

\section{2025/05/21}
\begin{itemize}
    \item LoRa: 通信間隔1分, 送信を終えるたびにDeepSleepモードになるように変更.
    \item 10:35送信開始. 電池残量4.18V ⇛ 翌日14:00まで正常に動作. (終了時電池残量2.95V(05/23/10:56))
\end{itemize}

\section{2025/05/23}
\begin{itemize}
    \item LoRa: 通信間隔2分, DeepSleep
    \item 11:11送信開始, 電池残量4.19V ⇛ 翌日14:33まで正常に動作.  (終了時電圧電池残量 2.52V(05/26/10:53))
\end{itemize}

\section{2025/05/26}
\begin{itemize}
    \item LoRa: 通信間隔10分, DeepSleep+センサ側もセンシングの度に停止.
    \item 15:20送信開始, 電池残量4.19V ⇛ 翌日20:01まで正常に動作
    \item センサ側の停止はデフォルトでreadするときのみになっているらしいので, 単純に10分にしたことになっている(かも)
\end{itemize}

\section{2025/06/02}
\begin{itemize}
    \item マイコン: SLEEPDEEPモードで実行. 間隔1分.
    \item 16:08送信開始, 電池残量4.2V ⇛ 
\end{itemize}


\section{2025/06/03}
\begin{itemize}
    \item マイコンは一旦忘れる. LoRa: 通信間隔1秒. 
    \item 18:15送信開始, 電池残量4.2V ⇛ 翌日02:05まで動作.
\end{itemize}

\section{2025/06/09}
\begin{itemize}
    \item パラメータを最適と思われるものに設定. 
    \item 1回の送信時間は60ms. 
    \item LoRa: 間隔1分
    \item 16:15送信開始, 電池残量4.19V ⇛ 翌日18:04まで動作.
\end{itemize}

\section{2025/06/10}
\begin{itemize}
    \item LoRa: 間隔2秒
    \item 18:33送信開始, 電池残量4.2V
\end{itemize}


\end{document}

